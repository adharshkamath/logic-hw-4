\documentclass[12pt,letterpaper, onecolumn]{exam}
\usepackage{amsmath}
\usepackage{amssymb}
\usepackage{listings}
\usepackage{hyperref}
\usepackage{xcolor}
\usepackage{bookmark}
\usepackage{graphicx}
\newcommand{\link}[1]{{\color{blue}\href{#1}{#1}}}
\usepackage{pythonhighlight}
\usepackage[a4paper,lmargin=30pt, rmargin=50pt, tmargin=0.65in]{geometry}  %For centering solution box
% \chead{\hline} % Un-comment to draw line below header
\thispagestyle{empty}   %For removing header/footer from page 1

\begin{document}

\begingroup
\centering
\LARGE CS 474\\
\large Homework 4 \\[0.5em]
\endgroup
\begingroup
\normalsize \quad\quad\quad Name: Adharsh Kamath \quad\quad\quad \quad\quad\quad \quad\quad\quad \quad\quad\quad \quad  NetID: ak128 \par\
\endgroup
\rule{17cm}{0.4pt}
\pointsdroppedatright   %Self-explanatory
\printanswers
\renewcommand{\solutiontitle}{\noindent\textbf{Soln:}\enspace}
\newcommand{\cheading}[1]{{\underline{\textit{#1}}}}

\renewcommand{\questionshook}{%
	\setlength{\leftmargin}{18pt}%
	\setlength{\labelwidth}{-\labelsep}%
}
\qformat{\underline{Problem \thequestion}}
\begin{questions}
	\question[]
	\solutiontitle
	A group is defined as a set S, along with a binary operation $ \cdot : S \times S \rightarrow S $ and axioms:
	\begin{enumerate}
		\item $\forall a \:,  b \:, c \:.\: (a \cdot b) \cdot c = a \cdot (b \cdot c)$
		\item $\forall a \:.\: ((a \cdot e) = a \land (e \cdot a) = a) $
		\item $\forall a \: \exists b \:.\: ((b \cdot a) = e \land (a \cdot b) = e)$
	\end{enumerate}

	where $ e $  is the special constant that is the identity element of the group.

	Task 1: 
	\begin{align*}
		G : \: \forall e' \:.\: ((\forall a \:.\: ((e \cdot a) = a \: \land (e' \cdot a) = a)) \implies (e = e'))
	\end{align*}

	The first two axioms are in prenex normal form, and do not contain any existential quantifiers.
	Skolemizing the third axiom (by replacing $ \exists b $ with a function $ f(a) $), we get:
	\begin{align*}
		\forall a \:.\: (a \cdot f(a)) = e \land (a \cdot f(a)) = e
	\end{align*}

	Simplifying the goal $G$, to prenex normal form:
	\begin{align*}
		\forall e' \: \forall a \:.\: ((a \cdot e') = a \: \land (e' \cdot a) = a) \implies (e = e') \\
		\equiv \forall e' \: \neg(\forall a \:.\: ((a \cdot e') = a \: \land (e' \cdot a) = a)) \lor (e = e') \\
		\equiv \forall e' \: (\exists a \:.\: \neg((a \cdot e') = a \: \land (e' \cdot a) = a)) \lor (e = e')
	\end{align*}

	To show the validity of the goal, we need to show that the negation of the goal is unsatisfiable along with all the axioms.

	\begin{align*}
		\neg G &\equiv \neg (\forall e' \: (\exists a \:.\: \neg((a \cdot e') = a \: \land (e' \cdot a) = a)) \lor (e = e')) \\
		&\equiv \exists e' \: \neg (\exists a \:.\: \neg((a \cdot e') = a \: \land (e' \cdot a) = a)) \land \neg (e = e') \\
		&\equiv \exists e' \: (\forall a \:.\: ((a \cdot e') = a \: \land (e' \cdot a) = a)) \land \neg (e = e')
	\end{align*}

	Skolemizing (with a constant $ e'' $ since $\exists e'$ is the outermost existential quantifier), we get:
	\begin{align*}
		\forall a \:.\: ((a \cdot e'') = a \: \land (e'' \cdot a) = a \land \neg (e = e''))
	\end{align*}

	Combining all the axioms, we have:
	\begin{align*}
		\forall a \:, b \:, c \:.\: (a \cdot b) \cdot c = a \cdot (b \cdot c) \\
		\forall a \:.\: ((a \cdot e) = a \land (e \cdot a) = a) \\
		\forall a \:.\: ((a \cdot f(a)) = e \land (f(a) \cdot a) = e) \\
		\forall a \:.\: ((a \cdot e'') = a \: \land (e'' \cdot a) = a \land \neg (e = e''))
	\end{align*}

	Instantiating with depth-0 terms ($e, e''$), we get a large set of formulae:
	\begin{align*}
		(e \cdot e) \cdot e = e \cdot (e \cdot e) \\
		(e \cdot e) \cdot e'' = e \cdot (e \cdot e'') \\
		(e \cdot e'') \cdot e = e \cdot (e'' \cdot e) \\
		(e \cdot e'') \cdot e'' = e \cdot (e'' \cdot e'') \\
		(e'' \cdot e) \cdot e = e'' \cdot (e \cdot e) \\
		(e'' \cdot e) \cdot e'' = e'' \cdot (e \cdot e'') \\
		(e'' \cdot e'') \cdot e = e'' \cdot (e'' \cdot e) \\
		(e'' \cdot e'') \cdot e'' = e'' \cdot (e'' \cdot e'') \\
		(e \cdot e) = e \land (e \cdot e) = e \\
		(e'' \cdot e) = e \land (e \cdot e'') = e \\
		(e \cdot f(e)) = e \land (f(e) \cdot (e)) = e \\
		(e'' \cdot f(e'')) = e \land (f(e'') \cdot (e'')) = e \\
		(e \cdot e'') = e \land (e'' \cdot e) = e \land \neg (e = e'') \\
		(e'' \cdot e'') = e'' \land (e'' \cdot e'') = e'' \land \neg (e = e'')
	\end{align*}
	
	In this list,
	we can find two conjuncts that are contradictory:
	\begin{align*}
		(e \cdot e'') = e'' \land (e'' \cdot e) = e'' \\
		(e'' \cdot e) = e \land (e \cdot e'') = e \land \neg (e = e'')
	\end{align*}
	Hence, there exists no model that satisfies all the formulae.
	Therefore, the goal $G$ is valid.

	The corresponding Z3 program is at the following URL:
	\link(https://tinyurl.comabcd)
	

	Task 2:
	Given the skolemized set of axioms:
	\begin{align*}
		\forall a \:, b \:, c \:.\: (a \cdot b) \cdot c = a \cdot (b \cdot c) \\
		\forall a \:.\: ((a \cdot e) = a \land (e \cdot a) = a) \\
		\forall a \:.\: ((a \cdot f(a)) = e \land (a \cdot f(a)) = e) \\
	\end{align*}
	we can see that $f$ is the inverse function. We can simplify the goal to 
	\begin{align*}
		G : \: \forall a, b . (((a \cdot b = e) \land (b \cdot a = e)) \implies (b = f(a)) )
	\end{align*}
	Negating this goal gives us:
	\begin{align*}
		\neg G : \: \exists a, b . (((a \cdot b = e) \land (b \cdot a = e)) \land \neg(b = f(a)))
	\end{align*}
	Skolemizing (with constants $a', b'$) gives us:
	\begin{align*}
		((a' \cdot b' = e) \land (b' \cdot a' = e)) \land \neg(b' = f(a'))
	\end{align*}

	Instantiating the axioms with depth-0 terms ($a', b'$), we get a long list of formulae. In this list,
	we can find a conjunct that contradicts the above formula:
	\begin{align*}
		((a' \cdot f(a')) = e \land (f(a') \cdot a') = e)
	\end{align*}



	

    {\rule{17cm}{0.4pt}}

	\question[]
	\solutiontitle

	The given formula $ \varphi $ is
	\begin{align*}
		\varphi = y \le x \land x \le y \land f(y) = f(7) \land x \le 5
	\end{align*}


    {\rule{17cm}{0.4pt}}

	\question[]
	\solutiontitle

	(a) \\
	Given $ f: 2^{N} \rightarrow 2^{N} $ that is defined as 
	\begin{align*}
		f(S) = \{2\} \cup \{y | y = 2x, x \in S\}
	\end{align*}


    {\rule{17cm}{0.4pt}}

\end{questions}
\end{document}