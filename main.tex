\documentclass[12pt,letterpaper, onecolumn]{exam}
\usepackage{amsmath}
\usepackage{amssymb}
\usepackage{listings}
\usepackage{hyperref}
\usepackage{xcolor}
\usepackage{bookmark}
\usepackage{graphicx}
\newcommand{\link}[1]{{\color{blue}\href{#1}{#1}}}
\usepackage{pythonhighlight}
\usepackage[a4paper,lmargin=30pt, rmargin=50pt, tmargin=0.65in]{geometry}  %For centering solution box
% \chead{\hline} % Un-comment to draw line below header
\thispagestyle{empty}   %For removing header/footer from page 1

\begin{document}

\begingroup
\centering
\LARGE CS 474\\
\large Homework 4 \\[0.5em]
\endgroup
\begingroup
\normalsize \quad\quad\quad Name: Adharsh Kamath \quad\quad\quad \quad\quad\quad \quad\quad\quad \quad\quad\quad \quad  NetID: ak128 \par\
\endgroup
\rule{17cm}{0.4pt}
\pointsdroppedatright   %Self-explanatory
\printanswers
\renewcommand{\solutiontitle}{\noindent\textbf{Soln:}\enspace}
\newcommand{\cheading}[1]{{\underline{\textit{#1}}}}

\renewcommand{\questionshook}{%
	\setlength{\leftmargin}{18pt}%
	\setlength{\labelwidth}{-\labelsep}%
}
\qformat{\underline{Problem \thequestion}}
\begin{questions}
	\question[]
	\solutiontitle
	A group is defined as a set S, along with a binary operation $ \cdot : S \times S \rightarrow S $ and axioms:
	\begin{enumerate}
		\item $\forall a \:,  b \:, c \:.\: (a \cdot b) \cdot c = a \cdot (b \cdot c)$
		\item $\forall a \:.\: ((a \cdot e) = a \land (e \cdot a) = a) $
		\item $\forall a \: \exists b \:.\: ((b \cdot a) = e \land (a \cdot b) = e)$
	\end{enumerate}

	where $ e $  is the special constant that is the identity element of the group.

	Task 1: 
	\begin{align*}
		G : \: \forall e' \:.\: ((\forall a \:.\: ((e \cdot a) = a \: \land (e' \cdot a) = a)) \implies (e = e'))
	\end{align*}

	The first two axioms are in prenex normal form, and do not contain any existential quantifiers.
	Skolemizing the third axiom (by replacing $ \exists b $ with a function $ f(a) $), we get:
	\begin{align*}
		\forall a \:.\: (a \cdot f(a)) = e \land (a \cdot f(a)) = e
	\end{align*}

	Simplifying the goal $G$, to prenex normal form:
	\begin{align*}
		\forall e' \: \forall a \:.\: ((a \cdot e') = a \: \land (e' \cdot a) = a) \implies (e = e') \\
		\equiv \forall e' \: \neg(\forall a \:.\: ((a \cdot e') = a \: \land (e' \cdot a) = a)) \lor (e = e') \\
		\equiv \forall e' \: (\exists a \:.\: \neg((a \cdot e') = a \: \land (e' \cdot a) = a)) \lor (e = e')
	\end{align*}

	To show the validity of the goal, we need to show that the negation of the goal is unsatisfiable along with all the axioms.

	\begin{align*}
		\neg G &\equiv \neg (\forall e' \: (\exists a \:.\: \neg((a \cdot e') = a \: \land (e' \cdot a) = a)) \lor (e = e')) \\
		&\equiv \exists e' \: \neg (\exists a \:.\: \neg((a \cdot e') = a \: \land (e' \cdot a) = a)) \land \neg (e = e') \\
		&\equiv \exists e' \: (\forall a \:.\: ((a \cdot e') = a \: \land (e' \cdot a) = a)) \land \neg (e = e')
	\end{align*}

	Skolemizing (with a constant $ e'' $ since $\exists e'$ is the outermost existential quantifier), we get:
	\begin{align*}
		\forall a \:.\: ((a \cdot e'') = a \: \land (e'' \cdot a) = a \land \neg (e = e''))
	\end{align*}

	Combining all the axioms, we have:
	\begin{align*}
		\forall a \:, b \:, c \:.\: (a \cdot b) \cdot c = a \cdot (b \cdot c) \\
		\forall a \:.\: ((a \cdot e) = a \land (e \cdot a) = a) \\
		\forall a \:.\: ((a \cdot f(a)) = e \land (f(a) \cdot a) = e) \\
		\forall a \:.\: ((a \cdot e'') = a \: \land (e'' \cdot a) = a \land \neg (e = e''))
	\end{align*}

	Instantiating with depth-0 terms ($e, e''$), we get a large set (conjunction) of formulae :
	\begin{align*}
		(e \cdot e) \cdot e = e \cdot (e \cdot e) \\
		(e \cdot e) \cdot e'' = e \cdot (e \cdot e'') \\
		(e \cdot e'') \cdot e = e \cdot (e'' \cdot e) \\
		(e \cdot e'') \cdot e'' = e \cdot (e'' \cdot e'') \\
		(e'' \cdot e) \cdot e = e'' \cdot (e \cdot e) \\
		(e'' \cdot e) \cdot e'' = e'' \cdot (e \cdot e'') \\
		(e'' \cdot e'') \cdot e = e'' \cdot (e'' \cdot e) \\
		(e'' \cdot e'') \cdot e'' = e'' \cdot (e'' \cdot e'') \\
		(e \cdot e) = e \land (e \cdot e) = e \\
		(e'' \cdot e) = e \land (e \cdot e'') = e \\
		(e \cdot f(e)) = e \land (f(e) \cdot (e)) = e \\
		(e'' \cdot f(e'')) = e \land (f(e'') \cdot (e'')) = e \\
		(e \cdot e'') = e \land (e'' \cdot e) = e \land \neg (e = e'') \\
		(e'' \cdot e'') = e'' \land (e'' \cdot e'') = e'' \land \neg (e = e'')
	\end{align*}
	
	In this list,
	we can find two conjuncts that are contradictory:
	\begin{align*}
		(e \cdot e'') = e'' \land (e'' \cdot e) = e'' \\
		(e'' \cdot e) = e \land (e \cdot e'') = e \land \neg (e = e'')
	\end{align*}
	Hence, there exists no model that satisfies all the formulae.
	Therefore, the goal $G$ is valid.

	The corresponding Z3 program is at the following URL:
	\link{https://github.com/adharshkamath/logic-hw-4/blob/main/p1\_1.py} \\
	

	Task 2:
	Given the skolemized set of axioms:
	\begin{align*}
		\forall a \:, b \:, c \:.\: (a \cdot b) \cdot c = a \cdot (b \cdot c) \\
		\forall a \:.\: ((a \cdot e) = a \land (e \cdot a) = a) \\
		\forall a \:.\: ((f(a)\cdot a) = e \land (a \cdot f(a)) = e) \\
	\end{align*}
	we can see that $f$ is the inverse function. We can simplify the goal to 
	\begin{align*}
		G : \: \forall a, b . (((a \cdot b = e) \land (b \cdot a = e)) \implies (b = f(a)) )
	\end{align*}
	Negating this goal gives us:
	\begin{align*}
		\neg G : \: \exists a, b . (((a \cdot b = e) \land (b \cdot a = e)) \land \neg(b = f(a)))
	\end{align*}
	Skolemizing (with constants $a', b'$) gives us:
	\begin{align*}
		((a' \cdot b' = e) \land (b' \cdot a' = e)) \land \neg(b' = f(a'))
	\end{align*}

	Instantiating the axioms with depth-0 terms ($a', b', e$), we get a long list (conjunction) of formulae:
	\begin{align*}
		(a' \cdot a') \cdot a' = a' \cdot (a' \cdot a') \\
		(a' \cdot a') \cdot b' = a' \cdot (a' \cdot b') \\
		(a' \cdot b') \cdot a' = a' \cdot (b' \cdot a') \\
		(a' \cdot b') \cdot b' = a' \cdot (b' \cdot b') \\
		(b' \cdot a') \cdot a' = b' \cdot (a' \cdot a') \\
		(b' \cdot a') \cdot b' = b' \cdot (a' \cdot b') \\
		(b' \cdot b') \cdot a' = b' \cdot (b' \cdot a') \\
		(b' \cdot b') \cdot b' = b' \cdot (b' \cdot b') \\ %
		(a' \cdot a') \cdot e = a' \cdot (a' \cdot e) \\
		(a' \cdot e) \cdot a' = a' \cdot (e \cdot a') \\
		(a' \cdot e) \cdot e = a' \cdot (e \cdot e) \\
		(e \cdot a') \cdot a' = e \cdot (a' \cdot a') \\
		(e \cdot a') \cdot e = e \cdot (a' \cdot e) \\
		(e \cdot e) \cdot a' = e \cdot (e \cdot a') \\
		(e \cdot e) \cdot e = e \cdot (e \cdot e) \\  %
		(b' \cdot b') \cdot e = b' \cdot (b' \cdot e) \\
		(b' \cdot e) \cdot b' = b' \cdot (e \cdot b') \\
		(b' \cdot e) \cdot e = b' \cdot (e \cdot e) \\
		(e \cdot b') \cdot b' = e \cdot (b' \cdot b') \\
		(e \cdot b') \cdot e = e \cdot (b' \cdot e) \\
		(e \cdot e) \cdot b' = e \cdot (e \cdot b') \\ %
		(a' \cdot b') \cdot e = a' \cdot (b' \cdot e) \\
		(a' \cdot e) \cdot b' = a' \cdot (e \cdot b') \\
		(e \cdot a') \cdot b' = e \cdot (a' \cdot b') \\
		(e \cdot b') \cdot a' = e \cdot (b' \cdot a') \\
		(b' \cdot a') \cdot e = b' \cdot (a' \cdot e) \\
		(b' \cdot e) \cdot a' = b' \cdot (e \cdot a') \\ %
		(a' \cdot e) = a' \land (e \cdot a') = a' \\
		(b' \cdot e) = b' \land (e \cdot b') = b' \\
		(e \cdot e) = e \land (e \cdot e) = e \\ %
		(a' \cdot f(a')) = e \land (f(a') \cdot a') = e \\
		(b' \cdot f(b')) = e \land (f(b') \cdot b') = e \\
		(e \cdot f(e)) = e \land (f(e) \cdot e) = e
	\end{align*}

	In this list, we can find two conjuncts that are contradictory:
	\begin{align*}
		(a' \cdot f(a')) = e \land (f(a') \cdot a') = e \\
		(b' \cdot f(b')) = e \land (f(b') \cdot b') = e
	\end{align*}

	The Z3 program for this task is at the following URL:
	\link{https://github.com/adharshkamath/logic-hw-4/blob/main/p1\_pt2.py}


    {\rule{17cm}{0.4pt}}

	\question[]
	\solutiontitle

	(a) \\
	The given formula $ \varphi $
	\begin{align*}
		\varphi = y \le x \land x \le y \land f(y) = f(7) \land x \le 5
	\end{align*}
	contains terms from $ T_{UIF} $ and $ T_{N} $ (theory of UIF and theory of natural numbers).

	Replacing $ f(7) $ with $ f(w) $, and adding $ w= 7 $ as an additional formula, we get:
	\begin{align*}
		\varphi' = y \le x \land x \le y \land x \le 5 \land w = 7 \land f(y) = f(w) 
	\end{align*} 

	The part of the formula in $ T_{UIF} $ :
	\begin{align*}
		F_{UIF} : f(y) = f(w)
	\end{align*}
	and the part of the formula in $ T_{N} $ :
	\begin{align*}
		F_{N} : y \le x \land x \le y \land x \le 5 \land w = 7
	\end{align*}

	The conjunct $ F_{UIF} \land F_{N} $ is $(T_{UIF} \cup T_N)$-equisatisfiable to $ \varphi $.

	(b) \\
	The shared variables between $ F_{UIF} $ and $ F_{N} $ are:
	\begin{align*}
		V = \text{shared}(F_{UIF}, F_{N}) = \{y, w\}
	\end{align*}

	The two possible equivalence relations and their arrangements are:
	\begin{align*}
		E_1 = \{\{y\}, \{w\}\}, \quad \alpha_1(E_1, V) : y = w\\
		E_2 = \{\{y, w\}\}, \quad \alpha_2(E_2, V) : y \neq w
	\end{align*}

	Simplifying $F_N$, we can write:
	\begin{align*}
		F_N = y \le x \land x \le y \land x \le 5 \land w = 7 \\
			\equiv x = y \land x \le 5 \land w = 7
	\end{align*}

	We can see that $F_N$ is satisfiable over natural numbers only under the arrangement $\alpha_2$ (since $y \le 5$ and $w=7$).
	A satisfying model for $F_N$ is:  $ \{x = 5, y = 5, w = 7\} $ \\

	We can see that $F_{UIF}$ is satisfiable under this arrangement if $f$ maps all natural numbers to a constant (since we need to satisfy
	$ f(y) = f(w) \land y \neq w $).
	A satisfying model for $F_{UIF}$ is: 
	$ \{y = 5, w = 7, (f(t) = 0 \: \text{for any} \: t \in \mathbb{N})\} $.  \\

	From the above two models, we can derive a model for the original formula $ \varphi $.
	Since $y$ is the only shared variable that is present in the original formula, we need to make sure the 
	$y$ in the final model is consistent with above models we have found for the individual parts.

	A satisfying model for $ \varphi $ is: $ \{x = 5, y = 5, (f(t) = 0 \: \text{for any} \: t \in \mathbb{N}) \} $.

    {\rule{17cm}{0.4pt}}

	\question[]
	\solutiontitle

	(a) \\
	The following is the least fixed point of the given function $f$:
	\begin{align*}
		\text{lfp}(f) = \{0, 2, 4, ...\} = \{2n \:|\: n \in \mathbb{N}\}
	\end{align*}

	(b) \\
	$f$ is monotonic, if $A \subseteq B$ implies $f(A) \subseteq f(B)$.

	Consider two sets $A, B$ such that $A \subseteq B$.
	If $x$ is an element of $f(A)$, then that means $x \in X$ and there exists a $c$ such that $c \in \mathbb{N}$, $c$ is prime and $cx \in A$.
	Since $A \subseteq B$, $cx \in B$.
	This means, $x \in f(B)$ (since $c$ is prime and $cx \in B$).

	Therefore, $f(A) \subseteq f(B)$.

	Using the iterative method:
	\begin{align*}
		X_0 &= \emptyset \\
		X_1 &= f(X_0) = \{100\} \\
		X_2 &= f(X_1) = \{ 100, 50, 20 \} \\
		X_3 &= f(X_2) = \{ 100, 50, 20, 25, 10, 4 \} \\
		X_4 &= f(X_3) = \{ 100, 50, 20, 25, 10, 4, 5, 2 \} \\
		X_5 &= f(X_4) = \{ 100, 50, 20, 25, 10, 4, 5, 2 \} 
	\end{align*}

	We can see that $X_4 = X_5$, and hence the least fixed point is $ \{100, 50, 20, 25, 10, 4, 5, 2\} $.

	(c) \\
	Informally, $f$ "adds" 0 to a set, and removes 1 if it existed in the set.
	An operator is monotonic, if $A \subseteq B$ implies $f(A) \subseteq f(B)$.
	We can show monotonicity by that all elements of $f(A)$ are in $f(B)$ if all elements of $A$ are in $B$.

	Consider $x$ such that $x \in f(A)$. This means, $x \in A \cup \{0\}$ and $x \neq 1$.
	Consider the following cases for any such $x$:
	\begin{enumerate}
		\item $x \in A $: In this case, $x \in A$, and hence $x \in B$. Since we know $x \neq 1$ (because $x \in f(A)$), $x \in f(B)$.
		\item $x = 0 \land x \notin A$ : In this case, $x \in f(B)$ since 0 is always added by $f$.
	\end{enumerate}

	Therefore, $A \subseteq B$ implies $f(A) \subseteq f(B)$.

	The least fixed point of $f$ is $ \{0\} $. 

	(d) \\

	Informally, $f$ "adds" 1 to a set, adds the double of every input element to the set, and removes the odd numbers from the input set.
	Consider sets $A, B$ such that $A \subseteq B$, and $x \in f(A)$.
	Consider the following cases for any such $x$:
	\begin{enumerate}
		\item $x = 1$ : In this case, $x \in f(B)$ since 1 is always added by $f$.
		\item $x = 2n, n \in A$ : In this case, $n \in B$, and hence $2n \in f(B)$. Therefore $x \in f(B)$.
	\end{enumerate}
	Note that $x$ cannot be an odd number (other than 1), since $f$ removes all odd numbers from the input set.
	So the above cases are exhaustive.

	Therefore, $A \subseteq B$ implies $f(A) \subseteq f(B)$.

	The least fixed point of $f$ is $\{1\} \cup \{2x \: | \: x \in \mathbb{N}\}$


    {\rule{17cm}{0.4pt}}

\end{questions}
\end{document}